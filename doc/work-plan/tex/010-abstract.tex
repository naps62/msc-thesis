\begin{abstract}

High performance computing has suffered several recent changes, with the evolution of Heterogeneous Platforms, where multiple devices with different architectures are able to share workload of increase performance.
Several frameworks have been in development, to aid the programmer in using these platforms, by dynamically distributing workload. One of these is the GAMA framework, designed to deal with irregular applications, by unifying the multiple execution and memory models of each device into a single, hardware independent model. GAMA handles the workload distribution across multiple computational resources, and attempts to balance them based on measurements from previous executions, with the main goal of maximizing efficiency. So far, this framework has only been tested for small benchmark kernels, for comparison with similar alternatives using standardized algorithms, but still lacks a deeper study, using a more realistic case study, to better understand its limitations.

The subject of this dissertation is to provide a more in-depth evaluation of GAMA, using the progressive photon mapping algorithm as a case study. The goal is to validate its effectiveness with a full sized application, and identify points of improvement, not only with the analysis of the case study, but also via comparison with similar frameworks. 
\end{abstract}
