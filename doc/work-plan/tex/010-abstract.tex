\begin{abstract}

High performance computing has suffered several recent changes, with the evolution of Heterogeneous Platforms, where multiple devices with different architectures and programming models are able to share the workload of an application to increase performance.
Several frameworks have been in development, which aid the programmer in taking advantage of the multiple resources available, by dinamically distribute the workload. One of this is the GAMA framework, designed to deal with irregular applications on heterogeneous platforms, by unifying their multiple execution and memory models into a single, hardware independent model. GAMA handles the distribution of the workload accross the multiple computational resources available, and attempts to balance them by using a heuristic based on the measurements of previous executions of each task, with the main goal of maximizing efficiency. This framework has only been tested for standalone kernels running independently, for comparison with similar alternatives using standardized, well studied methods.

However, for a better understanding of the capabilities and limitations provided by GAMA, a deeper study is required, using a larger case study, where several different tasks race must share resource usage. This can introduce new issues, or uncover different approaches for better handling the execution.

The subject of this dissertation is to provide a more in-depth and analysis of GAMA, using the Progressive Photon Mapping algorithm as a case study. The goal is to validate its effectiveness with a full sized application, and identify possible points of improvement, not only with the analysis of the results with the case study, but also via comparison with the results from similar frameworks. 
\end{abstract}
