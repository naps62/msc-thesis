\section{Context}
\todox{Context}

Current high performance computing platforms are composed of different architectures working together to deliver a higher computational power. These platforms, usually called Heterogeneous Platforms, typically consist of one or more multi-core CPU, and other types of devices, such as massively parallel architectures like Graphical Processing Units (GPUs), or event more specialized architectures like Field-Programmable Gate Arrays (FPGAs) or Digital Signal Processors (DSPs).

Using different devices to share the workload and achieve higher performance can be a challenging task, since each platform has very specific characteristics, and usually requires specific programming and memory models to get better performance. Algorithms often have to be adapted as well, to run efficiently on a specific platform. For some problems, one algorithm might be the most effecient on a given platform, but provide poor results on another.

%These problems present an onverview of the challenges associated with HetPlats. 
Due to these issues, enabling an application to take advantage of all the available resources of a HetPlat becomes a harder problem. Since each platform will perform some tasks better than others, a balance must be achieved between the tasks and the resources, assigning each available task to the platform that will most likely achieve better performance at any given moment.

To tackle this problem, several frameworks have been proposed which  attempt to hide some of this difficulties. Some of these include StarPU \cite{augonnet2011starpu}, Harmony \cite{diamos2008harmony} and \GAMA \cite{joao2012gama}. These frameworks attempt to address problems such as memory management, or the scheduling of the multiple tasks issued for execution.

Most of these frameworks however, provide mechanisms suited mostly for regular applications. When dealing with irregular applications, problems like scheduling become even more difficult, since memory usage patterns are less predictable. This directly affects the decisions of the scheduler, and as such must be taken into account by the performance model, as explained in \cite{artur2012gama}.

\subsection{GAMA}

The GAMA (GPU And Multicore Aware) framework is a tool that provides the mechanisms required to efficiently execute data-parallel applications in a Heterogeneous Platform. It works as an abstraction layer above the hardware, moving the programmer away from the task of scheduling the workload across the multiple resources available. In GAMA, an application is a collection of ordered jobs, each of which is composed of several tasks that are applied to different data sets. All memory is abstracted into a global address space, working as a distributed shared memory system \cite{ricardo2012gama}

