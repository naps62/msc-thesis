\documentclass[main.tex]{subfiles}

\begin{document}

\subsection{Heterogeneous Platforms}

Initial programming methodologies for \hetplats consists mostly on programming the main application structure to a regular \cpu core, and offload some of the heavily data-parallel work to an accelerator, usually a \gpu. This often is coupled with manual testing to assert whether or not executing the task on the accelerator device, along with the required memory transactions, are actually beneficial to the overall performance. 
With the increased acceptance of accelerators as co-processors, development efforts have been made towards making this development process much easier, and thus produce more, and better applications.

To address this problem, several frameworks have been proposed in the last recent years. These frameworks are usually targeted specifically at \acp{HetPlat}, and are designed with a focus on their specific scheduling issues, which are considered a key issue. These frameworks include StartPU \cite{augonnet2011starpu}, Harmony \cite{diamos2008harmony}, and GAMA. \todo{a cena coreana que o zuca tinha dito a uns tempos}

These frameworks attempt to provide a bridge for programmers do develop applications suited to \hetplat, by addressing problems such as memory management or the scheduling of multiple tasks issued for execution. Most of them however, provide mechanisms suited mostly for regular applications. When dealing with irregular applications, problems like scheduling become even more difficult, in particular because execution times and memory usage patterns are less predictable. This directly affects the decisions of the scheduler, and as such must be taken into account by the performance model employed by the framework. Irregular applications are one of the targets of the \gama framework, which will be object of study throughout this dissertation, through the implementation of an irregular algorithm as a case study.

\subsubsection{A note on debugging}

It is hard enough to debug multi-threaded \cpu applications, and the addition of an extra device, with a different architecture, and possibly different mechanisms to handle debugging, introduces yet another layer of complexity to the development process. Conceptually, it is much harder for the programmer to keep track of the debugging process, since there is not as much control over what is happening. For instance, unless threads are fully synchronous, a breakpoint inserted at any given point of a parallel region might result in different threads stopping at arbitrarily different regions, affecting the global state of the application.

\subsubsection{Scheduling}

Perhaps the biggest issue about the usage of accelerators is the ability to fully take advantage of the resources. It is extremely important to not overload one processor with too much data while others remain idle, or to offload work to a different device when the execution time together with the cost of data transfers will result in no benefit, or possibly in even worse results. In irregular algorithms this is an even more difficult problem, due to the increased unpredictability.
For those reasons, in order for any \hetplat framework to properly manage the executions of multiple tasks, the scheduling policy is one the most important aspects. In \gama, the scheduling policy is also backed up by the global memory system, which handles memory utilization and data transfers \cite{thesisMariano12,artur2012gama,ricardo2012gama}.

\end{document}
