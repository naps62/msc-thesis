\documentclass[main.tex]{subfiles}
\begin{document}

\section{Case Study}

The problem chosen as a case-study for this work is the Progressive Photon Mapping Algorithm, first proposed in \cite{hachisuka2008progressive}, which is a global illumination method, and an evolution of classical Ray-Tracing and Path-Tracing. This family of methods has a wide variety of applications, with the most common one being the rendering of photo realistic scenarios, by calculating the global illumination to solve the rendering equation.

Rendering methods such as photon mapping are a common example of resource demanding, irregular applications, due to the high amount of information required to accurately describe a three-dimensional scene, and to realistically simulate all of its lighting effects. Therefore, it should serve as a suitable case study in the context of this dissertation.

\itododone{Introduce the photon mapping case study}

\end{document}
