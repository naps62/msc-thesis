\documentclass[main.tex]{subfiles}
\begin{document}

\subsection{Case Study}

The problem chosen as a case-study for this work is the progressive photon mapping algorithm, first proposed in \cite{hachisuka2008progressive}, which is a global illumination method, and an evolution of classical ray-tracing and path-tracing. This family of methods has a wide variety of applications, the most common one being the rendering of photo realistic scenes, by computing the global illumination to solve the rendering equation.

Rendering methods such as photon mapping are a common example of resource demanding, irregular applications, due to the high amount of data required to accurately describe a 3D scene, and to realistically simulate all of its lighting effects. Therefore, it should serve as a suitable case study in the context of this dissertation.

The details of this algorithm will be explained in more depth in \Cref{section:photon}

\end{document}
