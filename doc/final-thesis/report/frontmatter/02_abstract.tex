\documentclass[main.tex]{subfiles}

\begin{document}
\cleardoublepage
\phantomsection
\pdfbookmark{Abstract}{abstract}
\chapter*{\abstractname}

Recent evolution of high performance computing moved towards heterogeneous platforms: multiple devices with different architectures, characteristics and programming models, share application workloads. To aid the programmer to efficiently explore these heterogeneous platforms several frameworks have been under development. These dynamically manage the available computing resources through workload scheduling and data distribution, dealing with the inherent difficulties of different programming models and memory accesses. Among other frameworks, these include \gama and \starpu.

The \gama framework aims to unify the multiple execution and memory models of each different device in a computer system, into a single, hardware agnostic model. It was designed to efficiently manage resources with both regular and irregular applications, and currently only supports conventional \cpu devices and \cuda-enabled accelerators. \starpu has similar goals and features with a wider user based community, but it lacks a single programming model.

The main goal of this dissertation was an in-depth evaluation of a heterogeneous framework using a complex application as a case study. \gama provided the starting vehicle for training, while \starpu was the selected framework for a thorough evaluation. The progressive photon mapping irregular algorithm was the selected case study. The evaluation goal was to assert the \starpu effectiveness with a robust irregular application, and make a high-level comparison with the still under development \gama, to provide some guidelines for \gama improvement.

Results show that two main factors contribute to the performance of applications written with \starpu: the consideration of data transfers in the performance model, and chosen scheduler. The study also allowed some caveats to be found within the \starpu API. Although this have no effect on performance, they present a challenge for new coming developers. Both these analysis resulted in a better understanding of the framework, and a comparative analysis with \gama could be made, pointing out the aspects where \gama could be further improved upon.

\end{document}
