\documentclass[main.tex]{subfiles}
\begin{document}

\chapter*{Agradecimentos}

Ao meu orientador, prof. Alberto Proença, por uma excelente orientação, quer na disponibilidade completa durante todo o ano, quer no rigor exigido e análise ao trabalho efectuado. Ao meu co-orientador, prof. Luis Paulo Santos, pelos desafios lançados no íncio do trabalho, e pela disponibilidade que demonstrou sempre que solicitado.

Aos colegas do LabCG, que me acolheram durante este ano num ótimo ambiente de trabalho, que foi essencial para esta dissertação. Em especial ao Roberto Ribeiro, pela ajuda e discussões ao longo do ano, que foram uma grande contribuição para este trabalho.

Aos meus colegas e amigos André Pereira e Pedro Costa, por todas as discussões, ajuda mútua e companheirismo durante estes dois anos.

A todos os membros do CeSIUM, núcleo que me acolheu como uma segunda casa durante todo o meu percurso académico.

Ao grande grupo de amigos que formei nesta universidade, cuja lista é demasiado grande para enumerar aqui, que me acomparanham  nestes que foram os melhores anos da minha vida, e me deram apoio durante os momentos de maior aoerto. Sem eles este trabalho não teria sido possível.

Aos colegas e amigos da GroupBuddies, em especial ao Roberto Machado, pela ajuda e compreensão demonstrada sempre que a minha ausência foi necessária.

Por fim, um especial agradecimento á minhão mãe e ao meu imão, pelo suporte e compreensão dada a minha acrescida ausência durante o ano.

\clearpage

\begin{quote}
  Work funded by the Portuguese agency FCT, \textit{Fundação para a Ciência e Tecnologia}, under the program UT Austin | Portugal.
\end{quote}

\begin{center}
  \includegraphics[width=0.5\textwidth]{fct}

  \includegraphics[width=0.5\textwidth]{utaustin}
\end{center}

\end{document}
