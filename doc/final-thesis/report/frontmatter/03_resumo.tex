\documentclass[main.tex]{subfiles}

\selectlanguage{portuguese}
\begin{document}
\cleardoublepage
\phantomsection
\pdfbookmark{Resumo}{resumo}
\chapter*{\abstractname}
  \section*{Uma avaliação das frameworks GAMA/StarPU para Plataformas Heterogéneas: O algoritmo de Progressive Photon Mapping}

  TRADUZIR ISTO

  A recente evolução da computação de alto desempenho é em direção ao uso de plataformas heterogéneas: múltiplos dispositivos com diferentes arquiteturas, características e modelos de programação, partilhando a carga computacional das aplicações. De modo a ajudar o programador a explorar eficientemente estas plataformas, várias frameworks têm sido desenvolvidas. Estas frameworks gerem os recursos computacionais disponíveis, tratando das dificuldades inerentes dos diferentes modelos de programação e acessos à memória. Entre outras frameworks, estas incluem o \gama e o \starpu.

  O \gama tem o objetivo de unificar os múltiplos modelos de execução e memória de cada dispositivo diferente num sistema computacional, transformando-os num único modelo, independente do hardware utilizado. A framework foi desenhada de forma a gerir eficientemente os recursos, tanto para aplicações regulares como irregulares, e atualmente suporta apenas \cpus convencionais e aceleradores \cuda. O \starpu tem objetivos e funcionalidades idênticos, e também um

\end{document}
