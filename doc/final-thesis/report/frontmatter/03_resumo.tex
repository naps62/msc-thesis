\documentclass[main.tex]{subfiles}

\selectlanguage{portuguese}
\begin{document}
\cleardoublepage
\phantomsection
\pdfbookmark{Resumo}{resumo}
\chapter*{\abstractname}
  \section*{Uma avaliação das frameworks GAMA/StarPU para Plataformas Heterogéneas: O algoritmo de Progressive Photon Mapping}

  A área da computação de alto desempenho foi alvo de uma grande evolução nos últimos anos, com a acrescida utilização de plataformas heterogéneas, nas quais são utilizados vários dispositivos que diferem não só na arquitectura, mas também nas caracteristicas, e no modelo de programação, trabalhando em colaboração e partilhando o conjunto de tarefas de uma aplicação com o objectivo final de melhorar o desempenho global.

  Várias \textit{frameworks} têm sido desenvolvidas com o intuito de facilitar a programação direccionada a estas plataformas, através de escalonamento dinamico da carga de trabalho pelos dispositivos disponíveis, e lidando com os pormenores e dificuldades inerentes da utilização destes sistemas de maior complexidade.

  Nestas frameworks incluem-se o \gama e o \starpu. A primeira está a ser desenvolvida de forma a tratar aplicações irregulares, nas quais o escalonamento de tarefas e a gestão de recursos é uma tarefa mais complicada. O \gama propõe um modelo unificado de programação e memória, agnóstico ao modelo usado internamente por cada dispositivo. A framework gere a carga de trabalho das várias unidades de computação, e tenta balanceá-la com base numa política de escalonamento e em informação obtida automaticamente em tempo de execução. O \starpu tem objectivos e caracteristicas similares, mas coloca maior ênfase na latência introduzida por transferências de memória, encarando-as como o principal limitador de eficiência na utilização de aceleradores.

  Esta dissertação propõe-se a disponibilizar uma avaliação intensiva do \starpu, usando o algoritmo irregular de \textit{Progressive Photon Mapping} como caso de estudo. O principal objectivo é validar a eficácia da framework para uma aplicação robusta, e culminar numa comparação com o \gama, que ainda se encontra em desenvolvimento, de forma a identificar os pontos fracos de cada uma delas, e como poderá o \gama evoluir da melhor forma.

\end{document}
