\documentclass[main.tex]{subfiles}
\begin{document}

\chapter{Frameworks for Heterogeneous Platforms}

The challenge of dynamically scheduling workload of an application across an entire heterogeneous system is an ambitious one, and has been focus of more attention as \hetplats emerge as efficient solutions for high performance computing. Among the several frameworks that have been proposed to target this and other issues are \gama and \starpu, which are here presented in more detail.
While both of them have common points in their philosophy (\gama is to some degree inspired by \starpu), they use slightly different approaches to manage \hetplats.
\todo{falar aqui de mais uma ou duas?}\todo{intro this}

\subfile{tex/310_gama}
\subfile{tex/320_starpu}

\todo{LPS: seria desejável uma subsecção realçando as diferenças entre o GAMA e o STARPU. Alias eu juntaria os caps 3 e 4 num único e poria esta subsecção no fim deste capitulo.}


\end{document}
