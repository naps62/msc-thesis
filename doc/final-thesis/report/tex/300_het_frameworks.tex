\documentclass[main.tex]{subfiles}
\begin{document}

\chapter{Frameworks for Heterogeneous Platforms} \label{chapter:frameworks}

The challenge of dynamically scheduling workload of an application across an entire heterogeneous system is an ambitious one, and has been focus of more attention as \hetplats emerge as efficient solutions for high performance computing. Among the several frameworks that have been proposed to target this and other issues are \gama and \starpu, which are here presented in more detail.
While both of them have common points in their philosophy (\gama is to some degree inspired by \starpu), they use slightly different approaches to manage \hetplats.

\itodo{usually scheduling this involves history based sampling, and memory management. this adds some degree of complexity in trying to have the framework manage data, which makes it sometimes difficult to use existing code, and keep compatibility with existing libraries}

\itodo{na comparison, dizer que o starpu permite o uso de bibliotecas externas, já que os kernels sao implementados de raiz. no gama, isso é mais complicado}

\subfile{tex/310_gama}
\subfile{tex/320_starpu}
\subfile{tex/330_comparison}

\end{document}
