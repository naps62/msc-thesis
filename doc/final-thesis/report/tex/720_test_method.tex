\documentclass[main.tex]{subfiles}

\begin{document}

\section{Testing Methodology} \label{sec:results:method}

\itodo{proença: o cap 4 descreveu o alg até ao SPPMPA. Convinha agora indicar que tempos foram medidos no conjunto de todas as tasks que foram identificadas atrás, e quando se usa GPU que tempos foram contabilizados (por ex. o tempo de transferencias de dados) que se pretendem de facto medir p/ se fazer uma avaliação da framework}

For each measurement, only the time spent in the main rendering function was considered, discarding any input and output time spent by the program. A minimum of 10 executions were made for each measurement, for which the 3 best executions within at most $5\%$ of each other were considered. When comparing results, the average time for each iteration of the main loop of SPPMPA was the base value to use (with each test running at least 20 iterations).

Whenever \cuda was employed, the \cuda Occupancy Calculator\footnote{A spreadsheet by \nvidia that helps estimating the ideal block size for a given kernel}, as well as manual tuning, were used to find the correct block size used for each computational kernel.

\subsection{Input Scene}

For simplicity, input reading was left to the LuxRender library, relying on the existing structures and parser to read all the data to render a scene. This limits all testing to the available scenes shipped with LuxRender as samples. From these scenes, only three scenes were selected, namely, \textbf{kitchen}, \textbf{cornell} and \textbf{luxball}.

\end{document}
