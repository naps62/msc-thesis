\documentclass[main.tex]{subfiles}

\begin{document}

\section{Testing Methodology} \label{sec:results:method}

For each measurement, only the time spent in the main rendering function was considered, discarding any input and output time spent by the program. A minimum of 10 executions were made for each measurement, for which the 3 best executions within at most $5\%$ of each other were considered.

Whenever \cuda was employed, the Cuda Occupancy Calculator\footnote{A spreadsheet by \nvidia that helps estimating the ideal block size for a given kernel}, as well as manual tuning, were used to find the correct block size used for each computational kernel.

\subsection{Input Scene}

For simplicity, input reading was left to the LuxRender library, relying on the existing structures and parser to read all the information regarding a scene to render. This limits all testing to the available scenes shipped with LuxRender as samples. From these scenes, only three scenes were selected (namely, \textbf{kitchen}, \textbf{cornell} and \textbf{luxball}, to analyze any potential differences the input data might have in the algorithm's efficiency. However, the limited amount of scenes to choose from may also be a limiting factor of that same analyzis.

\end{document}
