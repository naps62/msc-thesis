\documentclass[main.tex]{subfiles}
\begin{document}

\section{Introduction}

\subsection{Contextualization}

\itodo{Contextualization - refs refs refs}
Heterogeneous platforms are increasingly popular for high performance computing, taking advantage of accelerator devices to provide higher performance at lower costs, in addition to the already powerful tradicional \cpus. These accelerators are not as general-purpose as a tradicional \cpu, but have characteristics that make them suitable to outperform them in more specific, usually highly parallel tasks, and as such are useful as co-processors that complement the work of conventional systems.

Accelerators first started to appear with \gpus, which gradually evolved from specific hardware dedicated to graphics rendering, to fully featured general programming devices, capable of massive data parallelism and performance, at lower power consumptions. As of 2012, over 50 of the \footnote{A list of the most powerful supercomputers in the world, updated twice a year (\url{http://www.top500.org/})}{TOP500's} list were powered by \gpus, which indicates an exponential growth in usage when compared to previous years. This increased usage is motivated by the effectiveness of these devices for general-purpose computing.

Other types of accelerators since emerged, like the recent Intel \mic architecture, and while all of them differ from the traditional \cpu architecture, they also differ between themselves, providing different hardware specifications, along with different memory and programming models. \todo{falar mais do mic aqui}

Development of applications targeting these devices tends to be harder, or at least different from conventional programming. One has to take into account the differences of the underlying architecture, as well as the programming model being used, in order to produce code that is not only correct, but also efficient. And efficiency for one device might have a different set of requirements or rules that are inadequate to a different device. As a result, developers code have to take into account the characteristics of each different device they are using within their applications, if they want to fully take advantage of them. Usually, the task of producing the most efficient code for a single platform is a time consuming task, and requires a deep knowledge of the architecture itself, In addition, the parallel nature of these accelerators introduces yet another difficulty layer for developers.

Each accelerator can also be programmed in a variety of ways, ranging from low level programming models such as \cuda to higher level libraries like \openmp or \openacc. Each of these provides a different method of writing parallel programs, and has a different level of abstraction about the underlying architecture.

The complexity increases even further when it is considered that multiple accelerators might be used simultaneously. This agravates the already existing problems about workload, scheduling, and communication.

\todo{falar aqui de que talvez os aceleradores nao sejam `a cena`?}

These different types of devices are most commonly used (in the context of general-computing) as accelerators, in a system where at least one \cpu manages the main execution flow, and delegates specific tasks to the remaining computing resources. A system that uses different computational units is commonly referred to as a heterogeneous platform, here referred to as a \hetplat.

\end{document}
