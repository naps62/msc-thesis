\documentclass[main.tex]{subfiles}

\begin{document}

\chapter{Technological Background} \label{chapter:back}


\section{Parallel Computing}

Traditional computer programs are written in a sequential manner. It is natural to think of an algorithm as a sequence of steps that can be serially performed to achieve a final result. This has actually been the most common programming paradigm since the early days of computing, and it synergizes well with single processor machines. Optimizations related to instruction-level parallelism, such as out-of-order execution, pipelining, branch prediction or superscalarity, were mostly handled by the compiler or the hardware itself, and transparent to the programmer. Vector processing was also a key performance factor, allowing the same instruction to be applied to a vector of data simultaneously, rather than one at a time (commonly referred to as \ac{SIMD}).

But in the beginning of the \rom{21} century, the development of computational chips shifted from a single faster core perspective, to a multi core one. The evolution of single-core processors was already reaching its peak, and was slowing down due to the increasing difficulty in reducing transistor size or increasing clock frequencies, while introducing or aggravating other problems, such as heat dissipation, which becomes harder with the increased complexity of a chip. The solution was to move a multi-core perspective, coupling more cores in the same chip, to share the workload and allow overall computational capabilities to keep evolving.

This has allowed hardware development to keep in conformance with Moore's Law. And while it was a necessary step from a hardware's perspective, this has important implications in software development. In order for an application to take advantage of multi-core technology, it needs to be broken into smaller tasks, that can be independently executed, usually with some form of synchronization and communication between them. Writing parallel algorithms requires an adaptation to this new paradigm, as a sequential line of execution does not provide efficient results in platforms that support parallel execution of several threads or processes.

Writing parallel algorithms is generally not a trivial task compared their sequential counterpart. Several classes of problems are introduced to the programmer such as deadlocks, data races and memory consistency. Some of this problems may cause applications to behave unexpectedly under certain conditions. That, along with the fact that multiple execution lines are being processed in a sometimes very loose order, is also what makes the debugging of these applications much harder.

This is not helped by the fact that current development environments are still mostly unequipped to aid the programmer in such tasks. Support for debugging and profiling is still somewhat rudimentar in various cases, as should be expected from a paradigm that has not become mainstream until recent years.


\section{The \gpu as a Computing Accelerator}

With the increasing demand for highly data-parallel algorithms, and the growing amount of data to process, hardware development started shifting towards the goal of solving that problem. Initially, that started with the increased support for vector instructions in common \acsp{CPU}, and the \acs{SIMD} model. This allowed a single instruction to operate on a set of elements at once, effectively achieving a kind of parallelism which is extremely useful with highly data-parallel applications. Modern Intel processors, starting with the Sandy Bridge family, already support \ac{AVX}, an extension to the \textit{x86} instruction set allowing \acs{SIMD} instructions capable of handling 256 bits registers. This extension also introduces three-operand \acs{SIMD} instructions that allow more general $c = a + b$ operations to be handled with a single instruction, in addition to the previous instructions which only allowed two operands ($a = a + b$).

A similar and more flexible model is the usage of a single instruction across multiple threads running concurrently, called \ac{SIMT}. This is the programming model behind \gpus, which gradually started to gain more attention for their general purpose computing capabilities. Although the hardware of a \gpu is still tightly coupled with graphics processing and rendering, there have also been several advances in their usage as \gpgpus. \todo{ref aqui}

\todo{LPS: Estes 2 parágrafos como introdução para uma subsecção sobre GPUs não me parecem muito bem.
primeiro porque fala na vectorização e no numero de operandos ( porque não referir a MULTIPLY-ADD?) sem fazer a ponte para os GPUs; depois porque refere vagamente o SIMT mas sem dar qualquer pista sobre o que isto é.}


%%%%%%%%%%%%%%%%%%%%%%%%%%%%%%%%%%%%%%%
\subsection{Fermi Architecture}

The Fermi architecture was an important milestone of \gpus technology, as it was one of the first generations targeted directly towards \gpgpu and high performance computing, rather than purely graphics rendering. The first Fermi devices were released in 2010, and were the first NVidia products to include support for double precision floating point number, which was an important feature many fields that require increased precision. Fermi devices also included a GDDR5 memory controller with support for \ac{DMA} through the \acs{PCIe} bus, and up to 16 \acf{SM}, for a total of up to 512 \acs{CUDA} Cores.

\image[width=0.4\textwidth]{arch_fermi}{Overview of the Fermi Architecture}{fig:fermi}

This architecture is backed by a hardware-based thread scheduler, located within each \acs{SM}, that attempt to feed the execution unit with threads grouped in blocks of 32, or \textit{warps}. Since the scheduling is made directly via hardware, the switch between threads is nearly free, at least when compared with software scheduling on a \acs{CPU}. As a result, this strategy works better when the total amount of threads competing for resources is much higher than the amount of execution units, allowing for the latency of memory accesses to be hidden away by instantly scheduling a different \textit{warp}, effectively hiding memory latency while still keeping execution units busy. This is very different from \acs{CPU} scheduling policies, where switching between threads requires a context switch, which takes considerably longer, making that approach not as feasible as for a \gpu.



%%%%%%%%%%%%%%%%%%%%%%%%%%%%%%%%%%%%%%%%
\subsection{Kepler Architecture}

The follow-up generation to Fermi is in many ways similar to its predecessor. One of the most notorious change is the increase in the total amount of available CUDA cores, capable of going up to 2880 in high-end devices, due to the redesign of the \acl{SM}, now called \smx, each one with 192 \cuda Cores, although working at lower frequencies than before, due to the removal of shader frequency, a compromise to make room for the extra \cuda Cores. The entire chip now works based on the core frequency. Overall, individual core efficiency is lowered, but the global system becomes more efficient.

\image[width=0.4\textwidth]{arch_kepler}{Overview of the Kepler Architecture}{fig:kepler}

The programming model has been extended with the addition of dynamic parallelism, allowing \acs{CUDA} thread to spawn new threads, a feature not possible with previous versions. This is an important feature for irregular algorithms, such as photon mapping. It is also possible to invoke multiple kernels for a single \gpu, transparently dividing them between the available \smxs.

This shows a clear evolution over the previous generation, and a response to the increasing demand for highly parallel computational provided by \gpus.


\section{Intel MIC}
  todo{LPS. isto justificar-se-á? nao sao usados e o doc já vai longo}
\itodo{Xeon Phi}


\end{document}
