\documentclass[main.tex]{subfiles}

\begin{document}

\section{Testing Environment} \label{section:results:env}

All tests were performed within the \search\footnote{\url{http://search.di.uminho.pt}} cluster, particularly using the most recent generation of hardware, in the node 711, which is fully described in \cref{tab:node}.

\begin{table}[!htb]
    \centering
    \begin{tabular}{|ll|}
      \hline
      \cpu model: & \intel\xeon E5-2670 \\
      \# \cpus: & 2  \\
      \# Cores p/\cpu: & 8  \\
      \# Threads p/Core: & 2 \\
      Clock frequency: & 2.66 GHz \\
      \hline
      L1 cache: & 32 KB + 32 KB  \\
      L2 cache: & 256 KB \\
      L3 cache(shared): & 20 MB  \\
      RAM: & 64 GB  \\
      \hline
      \cuda Device 0: & Kepler K20m \\
      \cuda Device 1: & Tesla C20909 \\
      \hline
    \end{tabular}
  \caption{Hardware description of the \search computational node 711 \label{tab:node}}
\end{table}

%Later on, and partially due to availability issues, only the 711 was used for the final tests using the new implementation, both with and without \starpu.

All tests were compiled with GCC 4.6.2 (latest major version with full \cuda support), the boost library 1.49.0, and version 5.0 of the official \cuda compiler. Regarding \starpu, the latest version available, 1.1.0rc2, was used, as well as the hwloc 1.7 library for hardware topology, which \starpu internally uses.

\end{document}
