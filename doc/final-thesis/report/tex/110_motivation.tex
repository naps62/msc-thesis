\documentclass[main.tex]{subfiles}
\begin{document}

\section{Motivation \& Goals}

\hetplats, and consequently frameworks targeting them, can still be seen as a recent computing environment, especially when considering the volatility and constant evolution of computing systems. As such, there is still much to develop when it comes to the efficient usage of a \hetplat.

\gama is a recent framework that aims to provide the tools for developers to create dynamic applications, capable of efficiently running on these high performance computing platforms \cite{joao2012gama}
This framework is currently still under development, and supports only x86-64 \cpus and \cuda-capable \gpus. It is somewhat inspired in a similar framework, \starpu, but with an emphasis on the scheduling of irregular algorithms, a class which presents extra problems when dealing with workload scheduling. Although \gama has been deeply tested for a wide variety of kernels to validate the correctness and efficiency of its memory and execution model, it currently lacks a more intensive assessment, with a more robust and realistic application.
Small kernels have a wide range of applications, are deeply studied and optimized, and are a good source for an initial analysis on the performance results of the execution model. However, when considering a real and more resource intensive application, where possibly multiple tasks must share the available resources, other problems may arise. Therefore, a more realistic evaluation of the framework requires a more robust test case, as opposed to the currently used, more synthetic benchmarks.

The initial goal of this dissertation was to perform a quantitative and qualitative analysis of the \gama framework, applied to a large scale algorithm, in order to validate its effectiveness, and identify possible soft-spots, especially when compared to other similar frameworks. This would be done through the implementation of a case study, particularly the Progressive Photon Mapping Algorithm.

However, \gama still presents itself as an unfinished product. This raised some concerns during the initial stages of this work, as the current status of the framework could be too fragile, difficulting the implementation, as well as the accuracy of any study made on top of it, at least until a more stable version was ready.

 For this reasons, main focus was kept on \starpu, which is studied here through the implementation of a computationally intensive algorithm, providing the same analysis on the performance and usage of the framework, while serving as groundwork for future work to assert the effectiveness of \gama once possible. Additionally, a comparative analysis is also made, in order to establish where each framework excels, and what features a future release of \gama might require to be competitive against similar approaches with similar goals.

\starpu is an older project, and as a consequence, presents a more polished product, with the same overall goal of efficiently managing heterogeneous systems, but with a different philosophy and approach to the problem.

Overall, the work presented here consists on an analysis and comparison of \gama and \starpu, with the later being used for the implementation of an algorithm as a case study. Framework-less implementations were also developed to establish the baseline for any profiling analysis.

\end{document}
