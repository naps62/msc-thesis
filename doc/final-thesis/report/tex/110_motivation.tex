\documentclass[main.tex]{subfiles}
\begin{document}

\section{Motivation \& Goals}

\hetplats and associated development frameworks, can still be seen as a recent computing environment, especially when considering the volatility and constant evolution of computing systems. As such, there is still much to develop when it comes to the efficient usage of a \hetplat.

\gama is a recent framework under development at University of Minho and University of Texas at Austin, which aims to provide tools for developers to deploy dynamic applications, that efficiently run on these high performance computing platforms \cite{joao2012gama}.
This framework is still under development, and currently supports only x86-64 \cpus and \cuda-enabled \gpus. It is somewhat inspired in a similar framework, \starpu, but with an emphasis on the scheduling of irregular algorithms, a class which presents extra problems when dealing with workload scheduling. Although \gama has been deeply tested for a wide variety of kernels to validate the correctness and efficiency of its memory and execution model, it currently lacks a more intensive assessment, with a more robust and realistic application.
Small kernels have a wide range of applications, are deeply studied and optimized, and are a good source for an initial analysis on the performance results of the execution model. However, when considering a real and more resource intensive application, where possibly multiple tasks must share the available resources, other problems may arise. Therefore, a more realistic evaluation of the framework requires a richer set of robust test cases, to complement the performance evaluations made so far.

The initial goal of this dissertation was to perform a quantitative and qualitative analysis of the \gama framework, applied to a large scale algorithm, to validate its effectiveness, and identify possible soft-spots, especially when compared to other similar frameworks. This would be done through the implementation of a real case study, namely the progressive photon mapping algorithm, used in computer graphics.
However, a stable and reliable version of the \gama framework was not available during the time slot for this dissertation work and the qualitative evaluation of the framework was shifted to a competitor framework, \starpu.

\starpu is an older project, presenting a more polished product, with the same overall goal of efficiently managing heterogeneous systems, but with a different philosophy and approach to the problem.

\starpu is analysed here through the implementation of a computationally intensive algorithm, providing the same analysis on the performance and usage of the framework, while serving as groundwork for future work to assert the effectiveness of \gama. Additionally, a comparative analysis is also made, in order to establish where each framework excels, and what features a future release of \gama might require to be competitive against similar approaches with similar goals.

Overall, the work presented here consists on an analysis and a quantitative evaluation of \gama and \starpu, with the later being used for the implementation of a case study algorithm. Framework-less implementations were also developed to establish the baseline for profiling analysis. The analysis of \starpu through the case study helps in performing a comparison with \gama, both from the usability point of view, but also in terms of performance. Final conclusions indicate the downsides of each framework, as well as their strengths, and can serve as suggestions for future improvements of the \gama framework.

\end{document}
