\documentclass[main.tex]{subfiles}

\begin{document}

\chapter{Progressive Photon Mapping Algorithm as a Case Study} \label{chapter:case_study}

\subfile{tex/510_ray_tracing}
\subfile{tex/520_original}
\subfile{tex/530_progressive}
\subfile{tex/540_stochastic}
\subfile{tex/550_probabilistic}

\section{Summary}

Several techniques were presented, from the basic Photon Mapping algorithm, following a progressive approach that enables arbitrary accuracy without being memory limited. Further improvements include a stochastic version that enables additional accuracy in the final result, and a new formulation of the radius estimation that removes dependencies between iterations, effectively allowing their concurrency.

The final result of this is the algorithm Stochastic Progressive Photon Mapping with a Probabilistic Approach, here shortly called SPPMPA, which is the case study employed during the rest of this work.

\end{document}
